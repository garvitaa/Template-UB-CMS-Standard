\documentclass{beamer}

\mode<presentation>
{
  \usetheme{default}      % or try Darmstadt, Madrid, Warsaw, ...
  \usecolortheme{default} % or try albatross, beaver, crane, ...
  \usefonttheme{default}  % or try serif, structurebold, ...
  \setbeamertemplate{navigation symbols}{} % To remove the navigation symbols from the bottom of all slides uncomment this line
  \setbeamertemplate{footline}[page number] % To replace the footer line in all slides with a simple slide count uncomment this line
  \setbeamercolor{page number in head/foot}{fg=black}

} 
% For creating backup slides so that total # slides exclude backup slides, but the backup is still numbered
\newcommand{\backupbegin}{
   \newcounter{finalframe}
   \setcounter{finalframe}{\value{framenumber}}
}
\newcommand{\backupend}{
   \setcounter{framenumber}{\value{finalframe}}
}

% Packages
\usepackage[english]{babel}
\usepackage[utf8x]{inputenc}
\usepackage{graphicx} % Allows including images
\usepackage{booktabs} % Allows the use of \toprule, \midrule and \bottomrule in tables
\usepackage{tabto}
\usepackage[export]{adjustbox}
\usepackage{caption}
\usepackage{subcaption}
\usepackage{wrapfig}
\usepackage{multicol}
\usepackage{hyperref}
\usepackage{anyfontsize}
\usepackage{tcolorbox}
\usepackage{adjustbox}
\usepackage{tikzsymbols}
\usepackage{array}
\usepackage{amsmath}
\usepackage{hyperref}
\usepackage{listings}
\usepackage{xcolor}
\definecolor{olive}{rgb}{0.3, 0.4, .1}
\definecolor{fore}{RGB}{249,242,215}
\definecolor{back}{RGB}{51,51,51}
\definecolor{title}{RGB}{255,0,90}
\definecolor{dgreen}{rgb}{0.,0.6,0.}
\definecolor{gold}{rgb}{1.,0.84,0.}
\definecolor{JungleGreen}{cmyk}{0.99,0,0.52,0}
\definecolor{BlueGreen}{cmyk}{0.85,0,0.33,0}
\definecolor{RawSienna}{cmyk}{0,0.72,1,0.45}
\definecolor{Magenta}{cmyk}{0,1,0,0}
\usepackage{afterpage}

%---------------------------------------------------------------------------
%	TITLE PAGE
%---------------------------------------------------------------------------
\setbeamercolor{title}{fg=white}
\title[Your Short Title]{\textbf{YOUR PRESENTATION}} % The short title appears at the bottom of every slide, the full title is only on the title page
\setbeamercolor{author}{fg=white}
\author{You} % Your name
\setbeamercolor{institute}{fg=white}
\institute[University at Buffalo] % Your institution as it will appear on the bottom of every slide.
{
\textcolor{white}{\textit{username@buffalo.edu}} % Your email address
}
\setbeamercolor{date}{fg=white}
\date{\today} % Date, can be changed to a custom date

% To left align title page contents
\makeatletter
\setbeamertemplate{title page}[default][left,colsep=-4bp,rounded=true,shadow=\beamer@themerounded@shadow]
\makeatother

\begin{document}

{
%%%%%%%%% Uncomment the below line to use UB-CMS template
\usebackgroundtemplate{\includegraphics[width = 13.3cm, height = 9.6cm]{background/UB_background_title.pdf}}
%%%%%%%%% Uncomment the below line to use UB template
% \usebackgroundtemplate{\includegraphics[width = 13.3cm, height = 9.6cm]{background/UBTemplate_titlepage.pdf}}

\begin{frame}[plain]
    \titlepage
\end{frame}
  }

%---------------------------------------------------------------------------
%	PRESENTATION SLIDES
%---------------------------------------------------------------------------
% setting background
%%%%%%%%% Uncomment the two lines below to use UB-CMS template
\usebackgroundtemplate{\includegraphics[width = 12.8cm, height = 9.9cm]{background/background.pdf}}
\setbeamertemplate{frametitle}[default][center]% making all frame titles centered

%%%%%%%%% Uncomment the three lines below to use UB template
% \usebackgroundtemplate{\includegraphics[width = 12.8cm, height = 9.9cm]{background/UBTemplate_background.pdf}}
% \setbeamertemplate{frametitle}[default][left]% making all frame titles centered
% \setbeamercolor{frametitle}{fg=white}
%----------------------------------------------------------------------------------------
\begin{frame}{\textbf{Overview}}
\vspace{3.5mm}
Table of contents slide, comment this block out to remove it
\tableofcontents % Throughout your presentation, if you choose to use \section{} and \subsection{} commands, these will automatically be printed on this slide as an overview of your presentation
\end{frame}
%----------------------------------------------------------------------------------------

\section{Introduction}
%----------------------------------------------------------------------------------------
\begin{frame}{\textbf{Introduction}}
\vspace{3.5mm}

\begin{itemize}
  \item Your introduction goes here!
  \item Use \texttt{itemize} to organize your main points.
\end{itemize}

\vskip 1cm

\begin{block}{Examples}
Some examples of commonly used commands and features are included, to help you get started.
\end{block}

\end{frame}
%----------------------------------------------------------------------------------------
\section{Some \LaTeX{} Examples}

\subsection{Tables and Figures}

%----------------------------------------------------------------------------------------
\begin{frame}{\textbf{Tables and Figures}}
\vspace{3.5mm}
\begin{itemize}
\item Use \texttt{tabular} for basic tables --- see Table~\ref{tab:widgets}, for example.
\item You can upload a figure (JPEG, PNG or PDF) using the files menu. 
\item To include it in your document, use the \texttt{includegraphics} command (see the comment below in the source code).
\end{itemize}

% Commands to include a figure:
%\begin{figure}
%\includegraphics[width=\textwidth]{your-figure's-file-name}
%\caption{\label{fig:your-figure}Caption goes here.}
%\end{figure}

\begin{table}
\centering
\begin{tabular}{l|r}
Item & Quantity \\\hline
Widgets & 42 \\
Gadgets & 13
\end{tabular}
\caption{\label{tab:widgets}An example table.}
\end{table}

\end{frame}
%----------------------------------------------------------------------------------------
\subsection{Mathematics}
%----------------------------------------------------------------------------------------
\begin{frame}{\textbf{Readable Mathematics}}
\vspace{3.5mm}
Let $X_1, X_2, \ldots, X_n$ be a sequence of independent and identically distributed random variables with $\text{E}[X_i] = \mu$ and $\text{Var}[X_i] = \sigma^2 < \infty$, and let
$$S_n = \frac{X_1 + X_2 + \cdots + X_n}{n}
      = \frac{1}{n}\sum_{i}^{n} X_i$$
denote their mean. Then as $n$ approaches infinity, the random variables $\sqrt{n}(S_n - \mu)$ converge in distribution to a normal $\mathcal{N}(0, \sigma^2)$.

\end{frame}
%----------------------------------------------------------------------------------------

{\usebackgroundtemplate{}
\begin{frame}
\Huge{\centerline{The End}}
\end{frame}}
\appendix
\backupbegin

%----------------------------------------------------------------------------------------
% {\usebackgroundtemplate{}
% \begin{frame}
% \Huge{\centerline{Backup}}
% \end{frame}}
%----------------------------------------------------------------------------------------
\begin{frame}{\textbf{Extra Stuff}}
\vspace{3.5mm}
Something I wanted to keep in backup because I thought it was cool.
\end{frame}
%----------------------------------------------------------------------------------------
\backupend

\end{document}
